% LTeX: language=pt-BR

\section{Análise de Soluções Existentes}\label{sec:alt}

Nesta seção discutimos ferramentas de código aberto para identificação e gerenciamento de vulnerabilidades de \emph{software} (Seção~\ref{sec:alt.open}) bem como soluções de gerenciamento e priorização de vulnerabilidades (Seção~\ref{sec:alt.prio}).

\subsection{Ferramentas de Código Aberto}\label{sec:alt.open}

Nesta seção discutimos ferramentas de código aberto para levantamento de informações relacionadas a vulnerabilidades no contexto de uma entidade. Estas ferramentas são úteis para identificar e caracterizar vulnerabilidades em sistemas computacionais.

\begin{description}
    \item[Captura de tráfego de rede.] Ferramentas como \textsf{tcpdump}, WinDump, Wireshark, \textsf{nfdump}\footnote{\url{https://www.tcpdump.org}, \url{https://www.winpcap.org}, \url{https://www.wireshark.org} e \url{https://github.com/phaag/nfdump}} filtram, capturam e analisam pacotes de rede. Estas ferramentas proveem dados de baixo nível que são úteis para análises novas não disponíveis em outros sistemas.

    \item[Sistemas de detecção de intrusão de rede (NIDS).] Ferramentas como Nagios, Snort, Suricata e Zeek\footnote{\url{https://www.nagios.org}, \url{https://www.snort.org}, \url{https://suricata.io} e \url{https://zeek.org}} estendem ferramentas de captura de tráfego da rede com sistemas de verificação de regras ou detecção de anomalias para identificar tráfego malicioso ou dispositivos com comportamento suspeito em uma rede.

    \item[Sistemas de detecção de intrução de dispositivo (HIDS).] Ferramentas como OSSEC e Wazuh\footnote{\url{https://www.ossec.net} e \url{https://wazuh.com}} monitoram processos, tráfego de rede, arquivos e diretórios para identificar mudanças não autorizadas e comportamento anômalo.

    \item[Sistemas de varredura de rede.] Ferramentas como Masscan, Zmap, ZDNS e Zgrab\footnote{\url{https://github.com/robertdavidgraham/masscan} e \url{https://zmap.io}} realizam varreduras de portas e serviços em dispositivos de rede para identificar portas abertas e coletar informações de serviços em execução. Ferramentas como Nmap, OpenVAS e Nuclei\footnote{\url{https://nmap.org}, \url{https://www.openvas.org} e \url{https://projectdiscovery.io/nuclei}} realizam varreduras mais profundas, realizando sequências de requisições seguindo protocolos de aplicação e um conjunto de regras pré-definidas. Estas ferramentas analisam respostas recebidas para identificar propriedades de dispositivos e aplicações, incluindo a presença de vulnerabilidades de \emph{software}. Outras ferramentas fazem varreduras de aplicações e protocolos específicos, como o OWASP ZAP, Burp e Wapiti,\footnote{\url{https://www.zaproxy.org}, \url{https://portswigger.net/burp} e \url{https://wapiti-scanner.github.io}} que focam em aplicações Web; o SQLmap,\footnote{\url{https://sqlmap.org}} que foca em vulnerabilidades de \emph{software} relacionadas a bancos de dados; o Akto,\footnote{\url{https://www.akto.io}} que foca em vulnerabilidades em APIs Web.

    \item[Sistemas de verificação e teste de configuração.] Alguns sistemas tratam de testar a configuração de um software. Este tipo de ferramenta é útil na verificação de sistemas complexos. Exemplos de ferramentas incluem o
    BloodHound,\footnote{\url{https://github.com/BloodHoundAD/BloodHound}}, que enumera possíveis ataques contra o sistema de gerência de identidade e acesso Active Directory; o ScoutSuite,\footnote{\url{https://github.com/nccgroup/ScoutSuite}} que verifica a configuração de serviços em núvem; e o Lynis,\footnote{\url{https://cisofy.com/lynis}} que verifica a configuração de sistemas operacionais.

    \item[Descobrimento de informações.] Diversas ferramentas coletam informações adicionais sobre uma rede, o que é particularmente útil para identificar a superfície de vulnerabilidade de uma rede ou entidade. Exemplos de aplicações incluem o \textsf{netdiscover},\footnote{\url{https://github.com/netdiscover-scanner/netdiscover}} que utiliza análise passiva e sondagem ativa de pacotes ARP para descobrir dispositivos em uma rede; o \textsf{bettercap},\footnote{\url{https://www.bettercap.org}} que detecta dispositivos realizando varreduras WiFi, Bluetooth e IP; o Subfinder e o \textsf{sublist3r},\footnote{\url{https://github.com/projectdiscovery/subfinder} e \url{https://github.com/aboul3la/Sublist3r}} que descobrem subdomínios de um domínio; bem como o Amass e o \textsf{theHarverster},\footnote{\url{https://github.com/owasp-amass/amass} e \url{https://github.com/laramies/theharvester}} que descobrem informações sobre domínios a partir de dados públicos e motores de busca.

\end{description}

Uma enorme gama de ferramentas comerciais estão disponíveis para realização de análise de segurança. Muitas das ferramentas acima possuem versões comerciais que oferecem funcionalidades adicionais, como um conjunto mais completo de regras para detecção de vulnerabilidades, implementação de varreduras complementares, suporte técnico, atualizações automáticas e integração com outras ferramentas.

\subsection{Soluções para Gerência e Priorização de Vulnerabilidades}\label{sec:alt.prio}

Diversas empresas e projetos de código aberto proveem serviços e ferramentas para o monitoramento de vulnerabilidades. Nesta seção focamos em serviços de inteligência contra ameaças (\emph{threat intelligence}), discutindo funcionalidades comuns e abordagens de implementação.

\begin{description}
    \item[Soluções de código aberto.] Diversos projetos de \emph{software} livre disponibilizam sistemas para gerência de vulnerabilidades. Exemplos incluem o DefectDojo, ArcherySec, reNgine, Osmedeus, IVRE, Faraday e Crossfeed.\footnote{\url{https://www.defectdojo.com}, \url{https://www.archerysec.com}, \url{https://rengine.wiki/}, \url{https://www.osmedeus.org}, \url{https://ivre.rocks}, \url{https://faradaysec.com} e \url{https://docs.crossfeed.cyber.dhs.gov/}}  Estas soluções utilizam uma arquitetura modular (\emph{plugins}), onde cada módulo é responsável por realizar uma tarefa específica, como a varredura de vulnerabilidades em um sistema. Os módulos em geral executam ferramentas de código aberto, como as descritas na Seção~\ref{sec:alt.open}, processam a saída da ferramenta para converter em um formato padrão, armazenam as informações em um banco de dados, e proveem uma interface Web para visualização e inspeção das vulnerabilidades. Estes sistemas são úteis para organizar e consolidar informações sobre vulnerabilidades. Porém, atualmente nenhuma das soluções de código aberto que investigamos tem capacidade de priorizar vulnerabilidades.

    As abordagens de implementação destas ferramentas são diversas. Por exemplo, o DefectDojo agrega resultados de múltiplas ferramentas de monitoramento e armazena as vulnerabilidades encontradas em um banco de dados utilizando uma estrutura compartilhada (chamada \textsf{Finding}) que possui uma multitude de campos, a maioria deles pertinentes a um pequeno subconjunto dos módulos disponíveis, levando a desperdício de recursos. Outros sistemas utilizam estruturas específicas para cada ferramenta de monitoramento, o que evita desperdício de recursos, mas implica maior complexidade para manutenção e extensão.

    \item[Soluções comerciais.] A priorização de vulnerabilidades é um desafio complexo que requer a combinação de informações de diversas fontes e que pode ser abordada de diferentes formas. Não surpreendentemente, inúmeras ferramentas comerciais existem para detecção e gerência de vulnerabilidades. Este é um mercado em crescimento, com uma taxa de crescimento anual entre 2024--2028 estimada em 10.56\%.\footnote{\url{https://www.statista.com/outlook/tmo/cybersecurity/worldwide}} Muitas empresas apresentam soluções comerciais para a detecção, gerenciamento e priorização de vulnerabilidades, também chamadas de inteligência contra ameaças (\emph{threat intelligence}).
    Soluções comerciais são desenvolvidas por empresas de grande porte, como o IBM X-Force, Cisco Vulnerability Management, AWS GuardDuty e AT\&T OSSIM.\footnote{\url{https://www.ibm.com/x-force}, \url{https://www.cisco.com/c/en/us/products/security}, \url{https://aws.amazon.com/guardduty} e \url{https://cybersecurity.att.com/products/ossim}}
    \emph{Startups} de médio e pequeno porte como Tenable, Greenbone, Rapid7, Snyk, Symantec DeepSight\footnote{\url{https://www.tenable.com}, \url{https://www.greenbone.net}, \url{https://www.rapid7.com}, \url{https://snyk.io}, \url{https://docs.broadcom.com/doc/deepsight-intelligence-ds}} e várias outras\footnote{\url{https://www.recordedfuture.com}, \url{https://vulners.com}, \url{https://www.fortiguard.com}, \url{https://www.openwall.com}, \url{https://www.skyboxsecurity.com}, \url{https://www.trendmicro.com}, \url{https://www.flexera.com/products/software-vulnerability-manager}, \url{https://www.secureworks.com}, \url{https://www.reversinglabs.com}, \url{https://immunityinc.com} e \url{https://www.runzero.com}, \url{https://www.intruder.io}} também oferecem soluções comerciais para gerência de vulnerabilidades.

    Diversas soluções comerciais são ofertadas por empresas que mantém ferramentas e sistemas de código aberto. Exemplos incluem a Greenbone, que desenvolve o OpenVAS, a Tenable, que desenvolve o Nessus, e a Rapid7, que desenvolve o Metasploit. Estas empresas oferecem serviços de monitoramento e gerência de vulnerabilidades utilizando versões estendidas das ferramentas, geralmente capaz de detectar mais vulnerabilidades e com funcionalidades de gerência adicionais. Outras empresas proveem serviços de gerência de vulnerabilidade desenvolvidos sobre versões livres, por exemplo, o DefectDojo e Faraday possuem versões comerciais com funcionalidades adicionais como execução a partir de nuvem (em vez de auto-hospedado), integrações com sistemas de tíquete, controle de acesso, atualizações automáticas e agendamento de varredura não disponíveis nas versões de código aberto.

\end{description}

Abaixo discutimos três dimensões onde soluções de gerenciamento de vulnerabilidades apresentam diferentes abordagens e funcionalidades. No contexto comercial, estas diferenças são utilizadas para posicionar a solução desenvolvida no mercado relativo aos concorrentes. Infelizmente, uma comparação aprofundada de soluções comerciais é impossível, pois o acesso às ferramentas é restrito a clientes. Revisamos as informações disponíveis nas páginas destas soluções, mas as descrições nas páginas são vagas. A seguir apresentamos diferenças gerais das soluções comerciais e de código aberto avaliadas.

\begin{description}

    \item[Gerência de vulnerabilidades.] Muitas soluções incluem um sistema de gerência implementado como uma interface Web onde vulnerabilidades são registradas e acompanhadas até a resolução. As funcionalidades disponíveis no sistema de gerência varia entre aplicações. O ArcherySec, por exemplo, busca identificar vulnerabilidades encontradas por diferentes ferramentas comparando o campo de descrição. Esta é uma abordagem simples que pode levar a falsos positivos e falsos negativos, mas permite a consolidação de informações de sobre uma vulnerabilidade obtida de diferentes fontes. Soluções permitem o agrupamento de vulnerabilidades em diferentes granularidades, como dispositivos, máquinas virtuais, contêineres e aplicações. Por fim, soluções podem ter diferentes funcionalidades para acompanhar o processo de resolução de uma vulnerabilidade, por exemplo, armazenando anotações capturando informações sobre medidas tomadas por cada analista.

    \item[Pesquisa em detecção de vulnerabilidades.] Algumas empresas como Trendmicro, Tenable e Fortinet mantém equipes de analistas que trabalham efetivamente no descobrimento de novas vulnerabilidades (\emph{zero-day exploits}) para estender suas ferramentas. O esforço em pesquisar por novas vulnerabilidades também é realizado por pesquisadores e analistas independentes, que frequentemente contribuem extensões para soluções de monitoramento de código aberto.

    \item[Priorização de vulnerabilidades.]  Diversas soluções de gerenciamento permitem a priorização de vulnerabilidades. Abordagens incluem priorização manual pelo analista, priorização manual pela empresa em soluções comerciais (\emph{security advisories}) e priorização utilizando métricas de severidade como o (CVSS) ou a probabilidade de exploração (EPSS).

    Como identificado na Seção~\ref{sec:art.trends}, algumas soluções comerciais já disponibilizam assistentes de inteligência artificial. Considerando que não temos acesso às soluções propostas, é difícil avaliar as capacidades e eficácia destes assistentes. No momento, nenhum dos projetos de código aberto estudado apresentam assistentes de inteligência artificial.

    Soluções comerciais possuem serviços personalizados, que podem incluir a priorização de vulnerabilidades baseada em informações sobre uma entidade cliente. Porém, nosso estudo das soluções existentes não levantou nenhuma solução automática para priorização de vulnerabilidades considerando o contexto da empresa como pretendemos desenvolver.

\end{description}

Nesta seção não discutimos inúmeras ferramentas e serviços que abordam outros desafios relacionados a segurança. Por exemplo, a Proofpoint e a NetCraft\footnote{\url{https://www.proofpoint.com} e \url{https://www.netcraft.com}} oferecem soluções de segurança para identificar ameaças de \emph{phishing} e \emph{malware} em e-mails, documentos e sítios Web. A Cloudflare, a Akamai\footnote{\url{https://www.cloudflare.com} e \url{https://www.akamai.com}} e vários provedores de computação em núvem oferecem serviços de proteção contra ataques \emph{DDoS} e \emph{firewall} de aplicação. O BuiltWith\footnote{\url{https://builtwith.com}} identifica tecnologias utilizadas em sítios Web. Tripwire e CrowdStrike\footnote{\url{https://www.tripwire.com} e \url{https://www.crowdstrike.com}} monitoram dispositivos para detectar comportamento anônomalo ou mudanças não autorizadas. O Ghidra\footnote{\url{https://ghidra-sre.org/}} é uma ferramenta para engenharia reversa de código e realização de análises estáticas e dinâmicas. Estas e outras soluções são complementares aos serviços de inteligência discutidos nesta seção e contribuem para a segurança de sistemas computacionais, mas estão fora do escopo deste relatório.
